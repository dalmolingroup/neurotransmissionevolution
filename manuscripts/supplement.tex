% Options for packages loaded elsewhere
\PassOptionsToPackage{unicode}{hyperref}
\PassOptionsToPackage{hyphens}{url}
%
\documentclass[
]{article}
\usepackage{lmodern}
\usepackage{amssymb,amsmath}
\usepackage{ifxetex,ifluatex}
\ifnum 0\ifxetex 1\fi\ifluatex 1\fi=0 % if pdftex
  \usepackage[T1]{fontenc}
  \usepackage[utf8]{inputenc}
  \usepackage{textcomp} % provide euro and other symbols
\else % if luatex or xetex
  \usepackage{unicode-math}
  \defaultfontfeatures{Scale=MatchLowercase}
  \defaultfontfeatures[\rmfamily]{Ligatures=TeX,Scale=1}
\fi
% Use upquote if available, for straight quotes in verbatim environments
\IfFileExists{upquote.sty}{\usepackage{upquote}}{}
\IfFileExists{microtype.sty}{% use microtype if available
  \usepackage[]{microtype}
  \UseMicrotypeSet[protrusion]{basicmath} % disable protrusion for tt fonts
}{}
\makeatletter
\@ifundefined{KOMAClassName}{% if non-KOMA class
  \IfFileExists{parskip.sty}{%
    \usepackage{parskip}
  }{% else
    \setlength{\parindent}{0pt}
    \setlength{\parskip}{6pt plus 2pt minus 1pt}}
}{% if KOMA class
  \KOMAoptions{parskip=half}}
\makeatother
\usepackage{xcolor}
\IfFileExists{xurl.sty}{\usepackage{xurl}}{} % add URL line breaks if available
\IfFileExists{bookmark.sty}{\usepackage{bookmark}}{\usepackage{hyperref}}
\hypersetup{
  pdftitle={Ionotropic receptors as the driving force behind human synapse establishment},
  pdfauthor={Lucas H. Viscardi; Danilo O. Imparato; Maria Cátira Bortolini; Rodrigo J. S. Dalmolin},
  pdfkeywords={Evolution, Nervous system},
  hidelinks,
  pdfcreator={LaTeX via pandoc}}
\urlstyle{same} % disable monospaced font for URLs
\usepackage[margin=1in]{geometry}
\usepackage{color}
\usepackage{fancyvrb}
\newcommand{\VerbBar}{|}
\newcommand{\VERB}{\Verb[commandchars=\\\{\}]}
\DefineVerbatimEnvironment{Highlighting}{Verbatim}{commandchars=\\\{\}}
% Add ',fontsize=\small' for more characters per line
\usepackage{framed}
\definecolor{shadecolor}{RGB}{48,48,48}
\newenvironment{Shaded}{\begin{snugshade}}{\end{snugshade}}
\newcommand{\AlertTok}[1]{\textcolor[rgb]{1.00,0.81,0.69}{#1}}
\newcommand{\AnnotationTok}[1]{\textcolor[rgb]{0.50,0.62,0.50}{\textbf{#1}}}
\newcommand{\AttributeTok}[1]{\textcolor[rgb]{0.80,0.80,0.80}{#1}}
\newcommand{\BaseNTok}[1]{\textcolor[rgb]{0.86,0.64,0.64}{#1}}
\newcommand{\BuiltInTok}[1]{\textcolor[rgb]{0.80,0.80,0.80}{#1}}
\newcommand{\CharTok}[1]{\textcolor[rgb]{0.86,0.64,0.64}{#1}}
\newcommand{\CommentTok}[1]{\textcolor[rgb]{0.50,0.62,0.50}{#1}}
\newcommand{\CommentVarTok}[1]{\textcolor[rgb]{0.50,0.62,0.50}{\textbf{#1}}}
\newcommand{\ConstantTok}[1]{\textcolor[rgb]{0.86,0.64,0.64}{\textbf{#1}}}
\newcommand{\ControlFlowTok}[1]{\textcolor[rgb]{0.94,0.87,0.69}{#1}}
\newcommand{\DataTypeTok}[1]{\textcolor[rgb]{0.87,0.87,0.75}{#1}}
\newcommand{\DecValTok}[1]{\textcolor[rgb]{0.86,0.86,0.80}{#1}}
\newcommand{\DocumentationTok}[1]{\textcolor[rgb]{0.50,0.62,0.50}{#1}}
\newcommand{\ErrorTok}[1]{\textcolor[rgb]{0.76,0.75,0.62}{#1}}
\newcommand{\ExtensionTok}[1]{\textcolor[rgb]{0.80,0.80,0.80}{#1}}
\newcommand{\FloatTok}[1]{\textcolor[rgb]{0.75,0.75,0.82}{#1}}
\newcommand{\FunctionTok}[1]{\textcolor[rgb]{0.94,0.94,0.56}{#1}}
\newcommand{\ImportTok}[1]{\textcolor[rgb]{0.80,0.80,0.80}{#1}}
\newcommand{\InformationTok}[1]{\textcolor[rgb]{0.50,0.62,0.50}{\textbf{#1}}}
\newcommand{\KeywordTok}[1]{\textcolor[rgb]{0.94,0.87,0.69}{#1}}
\newcommand{\NormalTok}[1]{\textcolor[rgb]{0.80,0.80,0.80}{#1}}
\newcommand{\OperatorTok}[1]{\textcolor[rgb]{0.94,0.94,0.82}{#1}}
\newcommand{\OtherTok}[1]{\textcolor[rgb]{0.94,0.94,0.56}{#1}}
\newcommand{\PreprocessorTok}[1]{\textcolor[rgb]{1.00,0.81,0.69}{\textbf{#1}}}
\newcommand{\RegionMarkerTok}[1]{\textcolor[rgb]{0.80,0.80,0.80}{#1}}
\newcommand{\SpecialCharTok}[1]{\textcolor[rgb]{0.86,0.64,0.64}{#1}}
\newcommand{\SpecialStringTok}[1]{\textcolor[rgb]{0.80,0.58,0.58}{#1}}
\newcommand{\StringTok}[1]{\textcolor[rgb]{0.80,0.58,0.58}{#1}}
\newcommand{\VariableTok}[1]{\textcolor[rgb]{0.80,0.80,0.80}{#1}}
\newcommand{\VerbatimStringTok}[1]{\textcolor[rgb]{0.80,0.58,0.58}{#1}}
\newcommand{\WarningTok}[1]{\textcolor[rgb]{0.50,0.62,0.50}{\textbf{#1}}}
\usepackage{graphicx,grffile}
\makeatletter
\def\maxwidth{\ifdim\Gin@nat@width>\linewidth\linewidth\else\Gin@nat@width\fi}
\def\maxheight{\ifdim\Gin@nat@height>\textheight\textheight\else\Gin@nat@height\fi}
\makeatother
% Scale images if necessary, so that they will not overflow the page
% margins by default, and it is still possible to overwrite the defaults
% using explicit options in \includegraphics[width, height, ...]{}
\setkeys{Gin}{width=\maxwidth,height=\maxheight,keepaspectratio}
% Set default figure placement to htbp
\makeatletter
\def\fps@figure{htbp}
\makeatother
\setlength{\emergencystretch}{3em} % prevent overfull lines
\providecommand{\tightlist}{%
  \setlength{\itemsep}{0pt}\setlength{\parskip}{0pt}}
\setcounter{secnumdepth}{-\maxdimen} % remove section numbering
\usepackage{booktabs}
\usepackage{longtable}
\usepackage{array}
\usepackage{multirow}
\usepackage{wrapfig}
\usepackage{float}
\usepackage{colortbl}
\usepackage{pdflscape}
\usepackage{tabu}
\usepackage{threeparttable}
\usepackage{threeparttablex}
\usepackage[normalem]{ulem}
\usepackage{makecell}
\usepackage{xcolor}
\let\oldShaded\Shaded
\let\endoldShaded\endShaded
\renewenvironment{Shaded}{\scriptsize\oldShaded}{\endoldShaded}
\let\oldverbatim\verbatim
\let\endoldverbatim\endverbatim
\renewenvironment{verbatim}{\scriptsize\oldverbatim}{\endoldverbatim}
\usepackage{caption}
\captionsetup{justification=raggedright,singlelinecheck=false}
\captionsetup{margin={2pt,0pt}}

\title{Ionotropic receptors as the driving force behind human synapse
establishment}
\usepackage{etoolbox}
\makeatletter
\providecommand{\subtitle}[1]{% add subtitle to \maketitle
  \apptocmd{\@title}{\par {\large #1 \par}}{}{}
}
\makeatother
\subtitle{Supplementary Material}
\author{Lucas H. Viscardi \and Danilo O. Imparato \and Maria Cátira Bortolini \and Rodrigo J. S. Dalmolin}
\date{}

\begin{document}
\maketitle
\begin{abstract}
Model uncertainty and limited data are fundamental challenges to robust
management of human intervention in a natural system. These challenges
are acutely highlighted by concerns that many ecological systems may
contain tipping points, such as Allee population sizes. Before a
collapse, we do not know where the tipping points lie, if they exist at
all. Hence, we know neither a complete model of the system dynamics nor
do we have access to data in some large region of state-space where such
a tipping point might exist.
\end{abstract}

{
\setcounter{tocdepth}{3}
\tableofcontents
}
\hypertarget{project-structure}{%
\section{\texorpdfstring{\textbf{Project
structure}}{Project structure}}\label{project-structure}}

This is the title page

\hypertarget{preprocessing}{%
\section{\texorpdfstring{\textbf{Preprocessing}}{Preprocessing}}\label{preprocessing}}

This topic refers mainly to data wrangling done before the actual
analysis with the intent of making it simpler.

\hypertarget{eukaryota-species-tree}{%
\subsection{\texorpdfstring{\textbf{Eukaryota species
tree}}{Eukaryota species tree}}\label{eukaryota-species-tree}}

We opted to use the TimeTree database in order to obtain an standardized
Eukaryota species tree. However, some species were not present in it, so
we devised a way to fill them in based on NCBI Taxonomy data.

\hypertarget{ncbi-taxonomy-tree}{%
\subsubsection{\texorpdfstring{\textbf{NCBI Taxonomy
tree}}{NCBI Taxonomy tree}}\label{ncbi-taxonomy-tree}}

First we preprocess NCBI Taxonomy data to leave only STRING eukaryotes,
thus making the task easier. \hypertarget{downloading-data}{%
\paragraph{\texorpdfstring{\textbf{Downloading
data}}{Downloading data}}\label{downloading-data}}

\texttt{}

\begin{Shaded}
\begin{Highlighting}[]
\KeywordTok{download_if_missing}\NormalTok{(}\StringTok{"http://ftp.ncbi.nlm.nih.gov/pub/taxonomy/taxdump.tar.gz"}\NormalTok{)}
\KeywordTok{download_if_missing}\NormalTok{(}\StringTok{"stringdb-static.org/download/species.v11.0.txt"}\NormalTok{)}

\KeywordTok{untar}\NormalTok{(}\StringTok{"download/taxdump.tar.gz"}\NormalTok{, }\DataTypeTok{exdir =} \StringTok{"download/taxdump"}\NormalTok{)}
\end{Highlighting}
\end{Shaded}

\hypertarget{loading-data}{%
\paragraph{\texorpdfstring{\textbf{Loading
data}}{Loading data}}\label{loading-data}}

\texttt{}\\

\begin{table}[H]

\caption{\label{tab:string_species}Lists all organisms in STRING v11.}
\centering
\resizebox{\linewidth}{!}{
\begin{tabular}[t]{rllll>{\raggedright\arraybackslash}p{20em}}
\toprule
\multicolumn{6}{c}{\bgroup\fontsize{14}{16}\selectfont \cellcolor[HTML]{EEEEEE}{\ttfamily{\textbf{string\_species}}}\egroup{}} \\
\cmidrule(l{3pt}r{3pt}){1-6}
\# & Col. name & Col. type & Used? & Example & Description\\
\midrule
\rowcolor{gray!6}  1 & taxid & character & yes & 9606 & NCBI Taxonomy identifier\\
2 & string\_type & character & no & core & if the genome of this species is core or periphery\\
\rowcolor{gray!6}  3 & string\_name & character & yes & Homo sapiens & STRING species name\\
4 & ncbi\_official\_name & character & no & Homo sapiens & NCBI Taxonomy species name\\
\bottomrule
\multicolumn{6}{l}{\textbf{Location: } data-raw/download/species.v11.0.txt}\\
\multicolumn{6}{l}{\textbf{Source: } stringdb-static.org/download/species.v11.0.txt}\\
\end{tabular}}
\end{table}
\begin{table}[H]

\caption{\label{tab:ncbi_merged_ids}Links outdated taxon IDs to corresponding new ones.}
\centering
\resizebox{\linewidth}{!}{
\begin{tabular}[t]{rllll>{\raggedright\arraybackslash}p{20em}}
\toprule
\multicolumn{6}{c}{\bgroup\fontsize{14}{16}\selectfont \cellcolor[HTML]{EEEEEE}{\ttfamily{\textbf{ncbi\_merged\_ids}}}\egroup{}} \\
\cmidrule(l{3pt}r{3pt}){1-6}
\# & Col. name & Col. type & Used? & Example & Description\\
\midrule
\rowcolor{gray!6}  1 & taxid & character & yes & 9606 & id of node that has been merged\\
2 & new\_taxid & character & yes & core & id of node that is the result of merging\\
\bottomrule
\multicolumn{6}{l}{\textbf{Location: } data-raw/download/taxdump/merged.dmp}\\
\multicolumn{6}{l}{\textbf{Source: } ftp.ncbi.nlm.nih.gov/pub/taxonomy/taxdump.tar.gz}\\
\end{tabular}}
\end{table}
\begin{table}[H]

\caption{\label{tab:ncbi_edgelist}Represents taxonomy nodes.}
\centering
\resizebox{\linewidth}{!}{
\begin{tabular}[t]{rllll>{\raggedright\arraybackslash}p{20em}}
\toprule
\multicolumn{6}{c}{\bgroup\fontsize{14}{16}\selectfont \cellcolor[HTML]{EEEEEE}{\ttfamily{\textbf{ncbi\_edgelist}}}\egroup{}} \\
\cmidrule(l{3pt}r{3pt}){1-6}
\# & Col. name & Col. type & Used? & Example & Description\\
\midrule
\rowcolor{gray!6}  1 & taxid & character & yes & 2 & node id in NCBI taxonomy database\\
2 & parent\_taxid & character & yes & 131567 & parent node id in NCBI taxonomy database\\
\rowcolor{gray!6}  3 & rank & character & no & superkingdom & rank of this node\\
4 & ... & ... & no & ... & (too many unrelated fields)\\
\bottomrule
\multicolumn{6}{l}{\textbf{Location: } data-raw/download/taxdump/nodes.dmp}\\
\multicolumn{6}{l}{\textbf{Source: } ftp.ncbi.nlm.nih.gov/pub/taxonomy/taxdump.tar.gz}\\
\end{tabular}}
\end{table}
\begin{table}[H]

\caption{\label{tab:ncbi_taxon_names}Links taxon IDs to actual species names.}
\centering
\resizebox{\linewidth}{!}{
\begin{tabular}[t]{rllll>{\raggedright\arraybackslash}p{20em}}
\toprule
\multicolumn{6}{c}{\bgroup\fontsize{14}{16}\selectfont \cellcolor[HTML]{EEEEEE}{\ttfamily{\textbf{ncbi\_taxon\_names}}}\egroup{}} \\
\cmidrule(l{3pt}r{3pt}){1-6}
\# & Col. name & Col. type & Used? & Example & Description\\
\midrule
\rowcolor{gray!6}  1 & taxid & character & yes & 2 & the id of node associated with this name\\
2 & name & character & yes & Monera & name itself\\
\rowcolor{gray!6}  3 & unique\_name & character & no & Monera <bacteria> & the unique variant of this name if name not unique\\
4 & name\_class & character & yes & scientific name & type of name\\
\bottomrule
\multicolumn{6}{l}{\textbf{Location: } data-raw/download/taxdump/names.dmp}\\
\multicolumn{6}{l}{\textbf{Source: } ftp.ncbi.nlm.nih.gov/pub/taxonomy/taxdump.tar.gz}\\
\end{tabular}}
\end{table}

\texttt{}

\begin{Shaded}
\begin{Highlighting}[]
\NormalTok{string_species <-}\StringTok{ }\KeywordTok{read_tsv}\NormalTok{(}
  \StringTok{"download/species.v11.0.txt"}\NormalTok{,}
  \DataTypeTok{skip =} \DecValTok{1}\NormalTok{,}
  \DataTypeTok{col_names =} \KeywordTok{c}\NormalTok{(}
    \StringTok{"taxid"}\NormalTok{,}
    \StringTok{"string_type"}\NormalTok{,}
    \StringTok{"string_name"}\NormalTok{,}
    \StringTok{"ncbi_official_name"}
\NormalTok{  ),}
  \DataTypeTok{col_types =} \KeywordTok{cols_only}\NormalTok{(}
    \DataTypeTok{taxid =} \StringTok{"c"}\NormalTok{,}
    \DataTypeTok{string_name =} \StringTok{"c"}
\NormalTok{  )}
\NormalTok{)}

\CommentTok{# these .dmp files are very tricky to read}
\CommentTok{# the following read_delims are very hacky}
\NormalTok{ncbi_merged_ids <-}\StringTok{ }\KeywordTok{read_delim}\NormalTok{(}
  \StringTok{"download/taxdump/merged.dmp"}\NormalTok{,}
  \DataTypeTok{delim =} \StringTok{"|"}\NormalTok{,}
  \DataTypeTok{trim_ws =} \OtherTok{TRUE}\NormalTok{,}
  \DataTypeTok{col_names =} \KeywordTok{c}\NormalTok{(}\StringTok{"taxid"}\NormalTok{,}\StringTok{"new_taxid"}\NormalTok{),}
  \DataTypeTok{col_types =} \StringTok{"cc"}
\NormalTok{)}

\NormalTok{ncbi_edgelist <-}\StringTok{ }\KeywordTok{read_delim}\NormalTok{(}
  \StringTok{"download/taxdump/nodes.dmp"}\NormalTok{,}
  \DataTypeTok{skip =} \DecValTok{1}\NormalTok{,}
  \DataTypeTok{delim =} \StringTok{"|"}\NormalTok{,}
  \DataTypeTok{trim_ws =} \OtherTok{TRUE}\NormalTok{,}
  \DataTypeTok{col_names =} \KeywordTok{c}\NormalTok{(}\StringTok{"n1"}\NormalTok{,}\StringTok{"n2"}\NormalTok{),}
  \DataTypeTok{col_types =} \StringTok{"cc"}
\NormalTok{)}

\NormalTok{ncbi_taxon_names <-}\StringTok{ }\KeywordTok{read_delim}\NormalTok{(}
  \StringTok{"download/taxdump/names.dmp"}\NormalTok{,}
  \DataTypeTok{delim =} \StringTok{"|"}\NormalTok{,}
  \DataTypeTok{trim_ws =} \OtherTok{TRUE}\NormalTok{,}
  \DataTypeTok{col_names =} \KeywordTok{c}\NormalTok{(}\StringTok{"name"}\NormalTok{,}\StringTok{"ncbi_name"}\NormalTok{,}\StringTok{"type"}\NormalTok{),}
  \DataTypeTok{col_types =} \StringTok{"cc-c"}
\NormalTok{)}
\end{Highlighting}
\end{Shaded}

\hypertarget{updating-string-taxon-ids}{%
\paragraph{\texorpdfstring{\textbf{Updating STRING taxon
IDs}}{Updating STRING taxon IDs}}\label{updating-string-taxon-ids}}

\texttt{}\\
Some organisms taxon IDs are outdated in STRING. We must update them to
work with the most recent NCBI Taxonomy data.

\begin{Shaded}
\begin{Highlighting}[]
\NormalTok{string_species }\OperatorTok
\StringTok{  }\KeywordTok{left_join}\NormalTok{(ncbi_merged_ids) }\OperatorTok
\StringTok{  }\KeywordTok{mutate}\NormalTok{(}\DataTypeTok{new_taxid =} \KeywordTok{coalesce}\NormalTok{(new_taxid, taxid))}
\end{Highlighting}
\end{Shaded}

\hypertarget{creating-tree-graph}{%
\paragraph{\texorpdfstring{\textbf{Creating tree
graph}}{Creating tree graph}}\label{creating-tree-graph}}

\texttt{}\\
The first step is to create a directed graph representing the NCBI
Taxonomy tree.

\begin{Shaded}
\begin{Highlighting}[]
\CommentTok{# leaving only "scientific name" rows}
\NormalTok{ncbi_taxon_names }\OperatorTok
\StringTok{  }\KeywordTok{filter}\NormalTok{(type }\OperatorTok{==}\StringTok{ "scientific name"}\NormalTok{) }\OperatorTok
\StringTok{  }\KeywordTok{select}\NormalTok{(name, ncbi_name)}

\CommentTok{# finding Eukaryota taxid}
\NormalTok{eukaryota_taxon_id <-}\StringTok{ }\KeywordTok{subset}\NormalTok{(ncbi_taxon_names, ncbi_name }\OperatorTok{==}\StringTok{ "Eukaryota"}\NormalTok{, }\StringTok{"name"}\NormalTok{, }\DataTypeTok{drop =} \OtherTok{TRUE}\NormalTok{)}

\CommentTok{# creating graph}
\NormalTok{g <-}\StringTok{ }\KeywordTok{graph_from_data_frame}\NormalTok{(ncbi_edgelist[,}\DecValTok{2}\OperatorTok{:}\DecValTok{1}\NormalTok{], }\DataTypeTok{directed =} \OtherTok{TRUE}\NormalTok{, }\DataTypeTok{vertices =}\NormalTok{ ncbi_taxon_names)}

\CommentTok{# easing memory}
\KeywordTok{rm}\NormalTok{(ncbi_edgelist, ncbi_merged_ids)}
\end{Highlighting}
\end{Shaded}

\hypertarget{traversing-the-graph}{%
\paragraph{\texorpdfstring{\textbf{Traversing the
graph}}{Traversing the graph}}\label{traversing-the-graph}}

\texttt{}\\
The second step is to traverse the graph from the Eukaryota root node to
STRING species nodes. This automatically drops all non-eukaryotes and
results in a species tree representing only STRING eukaryotes (476).

\begin{Shaded}
\begin{Highlighting}[]
\NormalTok{eukaryote_root <-}\StringTok{ }\KeywordTok{V}\NormalTok{(g)[eukaryota_taxon_id]}
\NormalTok{eukaryote_leaves <-}\StringTok{ }\KeywordTok{V}\NormalTok{(g)[string_species[[}\StringTok{"new_taxid"}\NormalTok{]]]}

\CommentTok{# not_found <- subset(string_species, !new_taxid %in% ncbi_taxon_names$name)}

\NormalTok{eukaryote_paths <-}\StringTok{ }\KeywordTok{shortest_paths}\NormalTok{(g, }\DataTypeTok{from =}\NormalTok{ eukaryote_root, }\DataTypeTok{to =}\NormalTok{ eukaryote_leaves, }\DataTypeTok{mode =} \StringTok{"out"}\NormalTok{)}\OperatorTok{$}\NormalTok{vpath}

\NormalTok{eukaryote_vertices <-}\StringTok{ }\NormalTok{eukaryote_paths }\OperatorTok\StringTok{ }\NormalTok{unlist }\OperatorTok\StringTok{ }\NormalTok{unique}

\NormalTok{eukaryote_tree <-}\StringTok{ }\KeywordTok{induced_subgraph}\NormalTok{(g, eukaryote_vertices, }\DataTypeTok{impl =} \StringTok{"create_from_scratch"}\NormalTok{)}
\end{Highlighting}
\end{Shaded}

\hypertarget{saving}{%
\paragraph{\texorpdfstring{\textbf{Saving}}{Saving}}\label{saving}}

\texttt{}\\
Saving \texttt{ncbi\_tree} and \texttt{string\_eukaryotes} for package
use. These data files are documented by the package. We also create a
plain text file \texttt{476\_ncbi\_eukaryotes.txt} containing the
updated names of all 476 STRING eukaryotes. This file will be queried
against the TimeTree website.

\begin{Shaded}
\begin{Highlighting}[]
\NormalTok{ncbi_tree <-}\StringTok{ }\NormalTok{treeio}\OperatorTok{::}\KeywordTok{as.phylo}\NormalTok{(eukaryote_tree)}

\CommentTok{# plot(ncbi_tree %>% ape::ladderize(), type="cladogram")}

\NormalTok{string_eukaryotes <-}\StringTok{ }\NormalTok{string_species }\OperatorTok
\StringTok{  }\KeywordTok{filter}\NormalTok{(new_taxid }\OperatorTok\StringTok{ }\NormalTok{ncbi_tree}\OperatorTok{$}\NormalTok{tip.label) }\OperatorTok
\StringTok{  }\KeywordTok{inner_join}\NormalTok{(ncbi_taxon_names, }\DataTypeTok{by =} \KeywordTok{c}\NormalTok{(}\StringTok{"new_taxid"}\NormalTok{ =}\StringTok{ "name"}\NormalTok{))}

\KeywordTok{write}\NormalTok{(string_eukaryotes[[}\StringTok{"ncbi_name"}\NormalTok{]],}\StringTok{"476_ncbi_eukaryotes.txt"}\NormalTok{)}

\CommentTok{# usethis::use_data(ncbi_tree, overwrite = TRUE)}
\KeywordTok{write.tree}\NormalTok{(ncbi_tree, }\StringTok{"ncbi_tree.nwk"}\NormalTok{)}
\NormalTok{usethis}\OperatorTok{::}\KeywordTok{use_data}\NormalTok{(string_eukaryotes, }\DataTypeTok{overwrite =} \OtherTok{TRUE}\NormalTok{)}
\end{Highlighting}
\end{Shaded}

\begin{verbatim}
## <U+2714> Setting active project to 'C:/R/neuro'
## <U+2714> Saving 'string_eukaryotes' to 'data/string_eukaryotes.rda'
\end{verbatim}


\hypertarget{duplicated-genera}{%
\subsubsection{\texorpdfstring{\textbf{Duplicated
Genera}}{Duplicated Genera}}\label{duplicated-genera}}

Some species from different kingdoms may share the same genus name.
These genera must be noted down because one of the ways we fill in
missing species is by looking at genera names.
\hypertarget{loading-data}{%
\paragraph{\texorpdfstring{\textbf{Loading
data}}{Loading data}}\label{loading-data}}

\texttt{}\\
See \hyperref[tab:ncbi_edgelist]{Table 3} and
\hyperref[tab:ncbi_taxon_names]{Table 4}.

\begin{Shaded}
\begin{Highlighting}[]
\NormalTok{taxid_rank <-}\StringTok{ }\KeywordTok{read_delim}\NormalTok{(}
  \StringTok{"download/taxdump/nodes.dmp"}\NormalTok{,}
  \DataTypeTok{skip =} \DecValTok{1}\NormalTok{,}
  \DataTypeTok{delim =} \StringTok{"|"}\NormalTok{,}
  \DataTypeTok{trim_ws =} \OtherTok{TRUE}\NormalTok{,}
  \DataTypeTok{col_names =} \KeywordTok{c}\NormalTok{(}\StringTok{"taxid"}\NormalTok{,}\StringTok{"rank"}\NormalTok{),}
  \DataTypeTok{col_types =} \StringTok{"c-c"}
\NormalTok{)}

\NormalTok{ncbi_taxon_names <-}\StringTok{ }\KeywordTok{read_delim}\NormalTok{(}
  \StringTok{"download/taxdump/names.dmp"}\NormalTok{,}
  \DataTypeTok{delim =} \StringTok{"|"}\NormalTok{,}
  \DataTypeTok{trim_ws =} \OtherTok{TRUE}\NormalTok{,}
  \DataTypeTok{col_names =} \KeywordTok{c}\NormalTok{(}\StringTok{"taxid"}\NormalTok{,}\StringTok{"ncbi_name"}\NormalTok{,}\StringTok{"type"}\NormalTok{),}
  \DataTypeTok{col_types =} \StringTok{"cc-c"}
\NormalTok{)}
\end{Highlighting}
\end{Shaded}

\hypertarget{finding-duplicated-genera}{%
\paragraph{\texorpdfstring{\textbf{Finding duplicated
genera}}{Finding duplicated genera}}\label{finding-duplicated-genera}}

\texttt{}

\begin{Shaded}
\begin{Highlighting}[]
\CommentTok{# keeping genera nodes}
\NormalTok{taxid_rank }\OperatorTok\StringTok{ }\KeywordTok{filter}\NormalTok{(rank }\OperatorTok{==}\StringTok{ "genus"}\NormalTok{)}

\CommentTok{# keeping scientific names}
\NormalTok{ncbi_taxon_names }\OperatorTok
\StringTok{  }\KeywordTok{filter}\NormalTok{(type }\OperatorTok{==}\StringTok{ "scientific name"}\NormalTok{) }\OperatorTok
\StringTok{  }\KeywordTok{select}\NormalTok{(taxid, ncbi_name) }\OperatorTok
\StringTok{  }\KeywordTok{inner_join}\NormalTok{(taxid_rank)}

\CommentTok{# extracting and saving duplicated values}
\NormalTok{duplicated_genera <-}\StringTok{ }\NormalTok{ncbi_taxon_names }\OperatorTok
\StringTok{  }\KeywordTok{pull}\NormalTok{(ncbi_name) }\OperatorTok
\StringTok{  }\KeywordTok{extract}\NormalTok{(}\KeywordTok{duplicated}\NormalTok{(.)) }\OperatorTok
\StringTok{  }\KeywordTok{write}\NormalTok{(}\StringTok{"duplicated_genera.txt"}\NormalTok{)}
\end{Highlighting}
\end{Shaded}



\hypertarget{hybrid-tree}{%
\subsubsection{\texorpdfstring{\textbf{Hybrid
tree}}{Hybrid tree}}\label{hybrid-tree}}

Once we have both the NCBI eukaryotes tree and the list of duplicated
genera, we can start assembling the complete hybrid tree.
\input{tree_hybrid.tex}

\hypertarget{gene-selection-and-annotation}{%
\subsection{\texorpdfstring{\textbf{Gene selection and
annotation}}{Gene selection and annotation}}\label{gene-selection-and-annotation}}

The anchoring point for this study is basic annotation about genes and
the pathways in which they participate. This section describes the
process of structuring such data. In the end we will have a table to
which all kinds of additional data will be left joined into.
\input{kegg.tex}

\hypertarget{neuroexclusivity}{%
\subsection{\texorpdfstring{\textbf{Neuroexclusivity}}{Neuroexclusivity}}\label{neuroexclusivity}}

Explanation

\hypertarget{expression}{%
\subsubsection{\texorpdfstring{\textbf{Expression}}{Expression}}\label{expression}}

\hypertarget{pathways}{%
\subsubsection{\texorpdfstring{\textbf{Pathways}}{Pathways}}\label{pathways}}

\hypertarget{cog-data}{%
\subsection{\texorpdfstring{\textbf{COG
data}}{COG data}}\label{cog-data}}

\hypertarget{network}{%
\subsection{\texorpdfstring{\textbf{Network}}{Network}}\label{network}}

\hypertarget{analysis}{%
\section{\texorpdfstring{\textbf{Analysis}}{Analysis}}\label{analysis}}

Analysis

\begin{Shaded}
\begin{Highlighting}[]
\CommentTok{#leave this chunk}
\end{Highlighting}
\end{Shaded}

\end{document}
